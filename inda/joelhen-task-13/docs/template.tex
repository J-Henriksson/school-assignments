\documentclass{article}

\usepackage{fullpage}

\usepackage[utf8]{inputenc}
\usepackage{xcolor}

% Make math better
\usepackage{amsmath}
\usepackage{amssymb}
\usepackage{amsthm}

% Cross referencing and hyperreferencing
\usepackage{hyperref}
\usepackage{cleveref}

% Kill indentations in paragraphs
\setlength\parindent{0pt}

\title{Template for \LaTeX proof}
\date{}

\begin{document}

\maketitle

\textbf{Prove by induction} that the following statement is true for all positive integers.

\begin{align}
    \label{statement}
    \sum_{i=1}^{n} i^2 = \frac{n(n+1)(2n+1)}{6}
\end{align}

\begin{proof}

\textit{\\\\Base step} -- Show that \cref{statement} holds for the base case, when \dots.

\begin{align}
    \label{basestep}
    \dots
\end{align}

\textit{Inductive step} -- First, assume that statement is true for all $p$.

\begin{align}
    \label{assumption}
    \dots
\end{align}

Second, prove that the statement also holds true for $p+1$.

\begin{align}
    \label{proof}
    \dots
\end{align}

\textit{Conclusion} -- And we are done! We have proven by induction that the statement in~\cref{statement} is indeed true.

\end{proof}

\textbf{Tips on writing math in align mode}
\begin{itemize}
    \item Multiple statements can be included in one align environment. Force a new line with \textbackslash\textbackslash
    \item Every statement will get an index like (1). Use the \texttt{\textbackslash notag} command to suppress this when not needed
    \item Multiple statements can be aligned neatly using the \texttt{\&} character as an anchor point. Normally you want to place this beside the \texttt{=} symbol of each statement as follows: \texttt{\&=}
    \item If you need a particular symbol then refer to the \href{http://tug.ctan.org/info/undergradmath/undergradmath.pdf}{cheat sheet}.
\end{itemize}

For example, if $x=6$, what is $y$?
\begin{align}
    x + 3y &= 9 \notag\\
    (6) + 3y &= 9 \notag\\
    3y &= 9 - 6 \notag\\
    y &= \frac{9 - 6}{3} \notag\\
    y &= \frac{3}{3} \notag\\
    y &= 1 \notag
\end{align}

\end{document}
